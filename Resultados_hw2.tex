%--------------------------------------------------------------------
%--------------------------------------------------------------------
% Formato para los talleres del curso de Métodos Computacionales
% Universidad de los Andes
%--------------------------------------------------------------------
%--------------------------------------------------------------------

\documentclass[11pt,letterpaper]{exam}
\usepackage[utf8]{inputenc}
\usepackage[spanish]{babel}
\usepackage{graphicx}
\usepackage{tabularx}
\usepackage[absolute]{textpos} % Para poner una imagen en posiciones arbitrarias
\usepackage{multirow}
\usepackage{float}
\usepackage{hyperref}
%\decimalpoint

\begin{document}
\begin{center}
{\Large Métodos Computacionales} \\
\textsc{Tarea 2}\\
01-2019\\
\end{center}

\noindent

\section{Ejercicio 1: Fourier}

\subsection{ Señales graficadas}
\includegraphics{Senal.png}

\subsection{ Transformada Fourier de las señales}
\includegraphics{TransformadasSenales.png}

\subsection{ Espectograma de las señales}
\includegraphics{EspectrogramaSenales.png}

\subsection{ Señal sismica}
\includegraphics{SenalSismica.png}

\subsection{ Transformada Fourier señal sismica}
\includegraphics{TransformadaFourierSenalSismica.png}

\subsection{ Espectograma de las señal sismica}
\includegraphics{EspectrogramaTemblor.png}


%Analisis de resultados...

\end{document}
