%--------------------------------------------------------------------
%--------------------------------------------------------------------
% Formato para los talleres del curso de Métodos Computacionales
% Universidad de los Andes
%--------------------------------------------------------------------
%--------------------------------------------------------------------

\documentclass[11pt,letterpaper]{exam}
\usepackage[utf8]{inputenc}
\usepackage[spanish]{babel}
\usepackage{graphicx}
\usepackage{tabularx}
\usepackage[absolute]{textpos} % Para poner una imagen en posiciones arbitrarias
\usepackage{multirow}
\usepackage{float}
\usepackage{hyperref}
%\decimalpoint

\begin{document}
\begin{center}
{\Large Métodos Computacionales} \\
\textsc{Tarea 2}\\
01-2019\\
\end{center}

\noindent

\section{Ejercicio 1: Fourier}

\subsection{ Señales graficadas}
\includegraphics{Senal.png}
Las dos señales tiene comportamientos bastante distintos. Para la Señal Sumada parece que hubiese una oscilación dentro de una gran oscilación. Mientras que la Señal tiene un comportamiento más armonico con crecimientos progresivos para ambas partes de la gráfica. En la primera parte de la Señal se ve una frecuencia a la segunda parte. La segunda parte parece ser al menos doble de la anterior). Esto podria explicar porque en la Señal Sumada parece oscilar dentro de una gran oscilación. La sumatoria de las amplitudes podría anular alguna de ellas y a su vez crear amplitudes de 2*Señal.

\includegraphics{TransformadasSenales.png}
Ambas transformadas de Fourier muestran que las anteriores señales tienen dos frecuencias (una de ellas es el doble de la otra).  La transformada de la Señal sin sumar muestra un poco de ruido con amplitudes muy bajas. No sé como explicar eso ya esperaba que la superposición de las ondas creara nuevas frecuencias pero aparentemente no es así (revise el codigo y todo).

\subsection{ Espectograma de las señales}
\includegraphics{EspectrogramaSenales.png}
En el espectograma podemos ver algo muy parecido a lo que ya nos habia contado la Transformada de Fourier. Ambas señales tienen dos frecuencias y el lugar donde presentan mayores amplitudes (más energicas) es al rededor de 0.15s.


\subsection{ Señal sismica}
\includegraphics{SenalSismica.png}
Esta gráfica es bastante interesante. Muestra una onda posible onda P entre 0 a 400s con frecuencias de 2500Hz aprox. y una ligera traza de la onda S antes incrimentar su frecuencia considerablementa hasta casi 10000Hz.


\subsection{ Transformada Fourier señal sismica}
\includegraphics{TransformadaFourierSenalSismica.png}
La correspondiente Transformada de Fourier de nuestra señal sismica no es tan interesante. A pesar que se lograban identificar ondas P y S en la señal acá ya es mucho más complicado. Existe mucho ruido y seria muy complejo eliminarlo ya que tanto ondas P, S, Love y Rayleigh tienen frecuencias distintas. Es interesante ver el pico de amplitud que tiene antes de su decaimiento.


\subsection{ Espectograma de las señal sismica}
\includegraphics{EspectrogramaTemblor.png}
Este espectograma no dio. Es posible que se encuentre en las frecuencias muy bajas y que haya que limpiar la señal mucho para poder verlo (linea verde en Y=0).

\section{Ejercicio 2: ODE (Solucionado con el método LeapFrog )}
\subsection{ Para frecuencia de forzamiento = 1*w}
\includegraphics{amplitudes.png}
Las ecuaciones diferenciales de cada amplitud fueron resueltsa con el metood de Leap Frog. Para este forzamiento es interesante ver como todas las amplitudes varian de forma distinta y como la primera y la ultima son las que mas grandes amplitudes presentan. Parece que la ampitud 3 y 1 parecen estar en fase por tener sus más altas amplitudes en tiempos muy cercanos. Incrementan y decrecen juntas con un desfase muy pequeño.


\subsection{ Amplitud 1 para frecuencia de forzamiento = 1*w}
\includegraphics{amplitud1.png}
Para el caso de la amplitud 1 que es la del primer piso hay que resaltar que es una de la que más altas amplitudes presenta. Estos picos son bastante frecuentes (casi 5 veces en 5 minutos).


\subsection{ Amplitud 2 para frecuencia de forzamiento = 1*w}
\includegraphics{amplitud2.png}
Esta amplitud es la del segundo piso. Parece que es la que menos sufre ya que sus amplitudes son las mas bajas. Sus picos son mas constantes y se comporta inversamente proporcional a las otras dos amplitudes. Puede que las otras dos amplitudes se encuentren en fase y esta desfasada en al rededor de pi/2, por eso se comporten de manera contraria.


\subsection{ Amplitud 3 para frecuencia de forzamiento = 1*w}
\includegraphics{amplitud3.png}
Esta amplitud presenta las variaciones mas grandes en periodos de tiempo más cortos. Tiene además dos picos más picos que la amplitud uno. Se observa como hay momentos en que su amplitud se queda ligeramente constante. Puede deberse a la superposicion con la amplitud 2 que de cierta manera su movimiento.


\subsection{ Amplitudes maximas para una frecuencia de forzamiento entre 0.2*w hasta 3.0*w}
\includegraphics{amplitudesMax.png}
Esta grafica nos muestra frecuencias de forzamiento para las cuales el sistema se encuentra en resonancia y alcanza sus más altas amplitudes. Estos momentos son muy peligrosos ya que la fuerza es máximizada de forma natural para todas las amplitudes. Sobre esta gráfica se escogieron cuatro frecuencias de forzamiento para ver el comportamiento de las cuatro amplitudes.


\subsection{ Amplitudes para frecuencia de forzamiento = 0.4*w}
\includegraphics{amplitudes04.png}
Esta imagen se me hizo muy interesante. Si uno la animara podría ver que para esta frecuencia la parte más alta del edificio estaría teniendo las más altas amplitudes y las mas cerca al suelo las menos. Cómo si fuese un pendulo invertido. Para esta frecuencia ocurre una resonancia en la estructura lo que permite que este tipo de fenomenos ocurran: las ondas se encuentran oscilando en fase y sus amplitudes son consderablemente grandes. Esta onda no parece ser tan peligrosa por su baja frecuencia.

\subsection{ Amplitudes para frecuencia de forzamiento = 1.2*w}
\includegraphics{amplitudes12.png}
Este es otro momento donde las ondas se encuentran en resonacia. Sin embargo, en este caso las estructura se mueve como un pendulo normal. La parte de la estructura más cercana al piso (piso 1) es la que mas se mueve y los pisos mas altos son los que menos se mueven. Sus amplitudes no son tan altas pero parecen ser las mas altas para este piso. Es curioso porque pensaba que esto no podría pasar.


\subsection{ Amplitudes para frecuencia de forzamiento = 1.7*w}
\includegraphics{amplitudes17.png}
Para esta frecuencia de forzamiento la estructura se verá muy afectada. Principalmente porque alcanza sus maximas amplitudes 3 veces en 100s. Todos los pisos se moverian bastante pero el segundo sería el más afecto al tener las amplitudes más altas. Este es el único caso que la frecuencia permite que se alcance por lo menos 3 veces su maxima amplitud. Si se animase podría verse como el edificio se moveria abruptamente, parara, y se volviera a mover tres veces.


\subsection{ Amplitudes para frecuencia de forzamiento = 3.0*w}
\includegraphics{amplitudes30.png}
Las amplitudes son las más bajas, lo cual es bueno porque la estructura no se moveria tanto. La parte que más se mueve es la del primer piso y la frecuencia hace que sea un posible sismo que se sienta durante todos los 100 segundos sin parar.

\subsection{ Bono}
Se tomaron valores de estructuras del paper: Rodellar, J., Barbat, Á. H., Ryan, E. P., y Molinares, N. (1993). Comportamiento sísmico de edificios con un sistema no linel de control híbrido. Revista internacional de métodos numéricos para cálculo y diseño en ingeniería, 9(2), 201-220.
ISO 690. Las estructuras descritas en el paper son muy parecidas al edificio de mi casa. Utilizan un valor de 60000kg para cada piso lo que en mi caso daría una masa de 3x10E6 kg para todo el edificio y una rigidez de columnas de 4.5 a 9x10E8 N/m. Yo usaré una de 2x10E4 N/m ya que no encontre un buen parametro para comprar con mi casa y un poco mas realista ya que el edificio que probaron eran de hibrido antisismos y el mio no lo es. Además de un amorguamiento de 0.005 que es 10 veces menor al que ellos usaron para toda su estructura.
Mi objetivo es hallar la frecuencia a la cual mi edificio estaría en resonancia así que extendi la frecuencia de resonancia hasta 100000*w.
\includegraphics{amplitudesMaxBono.png}
En la grafica se puede observar los valores de W  el edificio con esos parametros descritos. Se encuentran varias pequeñas resonancias hasta una muy grande al rededor de 64000. Yo vivo en un primer piso, por lo cual en teoria sentiria más fuertes todas las pequeñas resonancias que ocurren hasta llegar al maximo. Sin embargo, sus amplitudes son muy pequeñas y por eso es muy probable que no se sienta nada. La frecuencia de las ondas es un factor que no sé como tener en cuenta en el momento de interpretar, ya que al ser tan constantes el movimiento podría debilitar la estructura y cambiar considerablemente su rigidez. Es posible que a pesar de que parezcan valores muy pequeños para la estructura sean muy grandes y pueda quebrarse muy facilmente.

\subsection{ Amplitudes frecuencia elegida}
\includegraphics{amplitudesBONO.png}
Las amplitudes son muy pequeñas. Esto muestra que apesar de que se encuentre en resonancia es muy probable que no lo sienta a pesar de que mi piso el que mas amplitud tenga. La simulación es muy incompleta y le faltan muchas variables a tener en cuenta. Sin embargo, es una buena aproximación a las sismulaciones de estructuras.


\subsection{ Análisis de resultados }
Los análisis de ondas a través de métodos computacionales y algoritmos nos pueden ofrecer información muy importante para diversas áreas. Una de las mas importantes, como lo mostrado en la tarea, es el análisis de estructuras que puede ayudar a crear de mejor manera estructuras antisismicas. Las sismulaciones pueden mostrar rangos en los que las estructuras estarían en un riesgo muy alto por hallarse en resonancia con el sismo. De igual forma, en el análisis de sismos en sí también es muy útil. Por medio de metodos de Fourier se mostró como identificar las distintas frecuencias que componen un sismo y ver cuales de ellas hacen parte del sismo en sí y cuales son solo ruido. A pensar de que las simulaciones no tienen los parametros adecuados estos pueden ser añadidos en futuro para elaborar simulaciones mas elaboradas y con datos más realistas.



\end{document}
